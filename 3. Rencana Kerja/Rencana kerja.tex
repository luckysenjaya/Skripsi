\documentclass[a4paper,twoside]{article}
\usepackage[T1]{fontenc}
\usepackage[bahasa]{babel}
\usepackage{graphicx}
\usepackage{graphics}
\usepackage{float}
\usepackage[cm]{fullpage}
\pagestyle{myheadings}
\usepackage{etoolbox}
\usepackage{setspace} 
\usepackage{lipsum} 
\setlength{\headsep}{30pt}
\usepackage[inner=2cm,outer=2.5cm,top=2.5cm,bottom=2cm]{geometry} %margin
% \pagestyle{empty}

\makeatletter
\renewcommand{\@maketitle} {\begin{center} {\LARGE \textbf{ \textsc{\@title}} \par} \bigskip {\large \textbf{\textsc{\@author}} }\end{center} }
\renewcommand{\thispagestyle}[1]{}
\markright{\textbf{\textsc{AIF401/AIF402 \textemdash Rencana Kerja Skripsi \textemdash Sem. Genap 2015/2016}}}

\onehalfspacing
 
\begin{document}

\title{\@judultopik}
\author{\nama \textendash \@npm} 

%tulis nama dan NPM anda di sini:
\newcommand{\nama}{Lucky Senjaya Darmawan}
\newcommand{\@npm}{2012730009}
\newcommand{\@judultopik}{Studi dan Integrasi Workflow menggunakan BPMS dan Sistem Email} % Judul/topik anda
\newcommand{\jumpemb}{1} % Jumlah pembimbing, 1 atau 2
\newcommand{\tanggal}{20/12/2016}

% Dokumen hasil template ini harus dicetak bolak-balik !!!!

\maketitle

\pagenumbering{arabic}

\section{Deskripsi}
\textit{Workflow} merupakan pemodelan proses bisnis yang dapat digambarkan sebagai \textit{flow map} atau BPMN \textit{(Business Process Model and Notation).} \textit{Workflow} ini dapat diotomasi menggunakan BPMS \textit{(Business Process Management System)}, seperti Camunda. Agar eksekusi \textit{workflow} lebih alamiah dengan model komunikasi organisasi saat ini, maka \textit{event} dapat dipropagasi dan diintegrasikan dengan sistem email.

\section{Rumusan Masalah}
\begin{itemize}
	\item Bagaimana memodelkan \textit{workflow} dengan BPMN?
	\item Event-event \textit{workflow} apa saja yang dapat dipropagasi ke sistem email?
	\item Bagaimana mekanisme propagasi dan integrasi \textit{workflow} dengan sistem email?
\end{itemize}
\section{Tujuan}
\begin{itemize}
	\item Memodelkan \textit{workflow} dengan BPMN.
	\item Mengidentifikasi event-event \textit{workflow} yang dapat dipropagasi ke sistem email.
	\item Menentukan mekanisme propagasi dan mengintegrasikan \textit{workflow} dengan sistem email.
\end{itemize}

\section{Deskripsi Perangkat Lunak}
Hasil akhir dari skripsi ini adalah integrasi \textit{workflow} yang diotomasi BPMS dengan sistem email. Ketika ada \textit{event} yang membutuhkan sistem email, maka perangkat lunak akan mengirimkan email atau membuka email yang masuk.

		

\section{Detail Pengerjaan Skripsi}
Bagian-bagian pekerjaan skripsi ini adalah sebagai berikut :
	\begin{enumerate}
		\item Studi tentang proses bisnis, \textit{workflow}, pemodelan BPMN, BPMS, sistem email.
		\item Memodelkan suatu proses bisnis tertentu menggunakan BPMN.
		\item Mengidentifikasikan event-event dari \textit{workflow} yang perlu dipropagasi via sistem email.
		\item Mengimplementasikan integrasi BPMS dengan sistem email.
		\item Membuat dokumen skripsi.
	\end{enumerate}

\section{Rencana Kerja}


\begin{center}
  \begin{tabular}{ | c | c | c | c | l |}
    \hline
    1*  & 2*(\%) & 3*(\%) & 4*(\%) &5*\\ \hline \hline
    1   & 25  &   & 25 &  \\ \hline
    2   & 15  &   & 15 &  \\ \hline
    3   & 15  &   & 15 &  \\ \hline
    4   & 25  &   & 25 &  \\ \hline
    5   & 20  &   & 20 &  \\ \hline
    Total  & 100  &   & 100 &  \\ \hline
                          \end{tabular}
\end{center}

Keterangan (*)\\
1 : Bagian pengerjaan Skripsi (nomor disesuaikan dengan detail pengerjaan di bagian 5)\\
2 : Persentase total \\
3 : Persentase yang akan diselesaikan di Skripsi 1 \\
4 : Persentase yang akan diselesaikan di Skripsi 2 \\
5 : Penjelasan singkat apa yang dilakukan di S1 (Skripsi 1) atau S2 (skripsi 2)

\vspace{1cm}
\centering Bandung, \tanggal\\
\vspace{2cm} \nama \\ 
\vspace{1cm}

Menyetujui, \\
\ifdefstring{\jumpemb}{2}{
\vspace{1.5cm}
\begin{centering} Menyetujui,\\ \end{centering} \vspace{0.75cm}
\begin{minipage}[b]{0.45\linewidth}
% \centering Bandung, \makebox[0.5cm]{\hrulefill}/\makebox[0.5cm]{\hrulefill}/2013 \\
\vspace{2cm} Nama: \makebox[3cm]{\hrulefill}\\ Pembimbing Utama
\end{minipage} \hspace{0.5cm}
\begin{minipage}[b]{0.45\linewidth}
% \centering Bandung, \makebox[0.5cm]{\hrulefill}/\makebox[0.5cm]{\hrulefill}/2013\\
\vspace{2cm} Nama: \makebox[3cm]{\hrulefill}\\ Pembimbing Pendamping
\end{minipage}
\vspace{0.5cm}
}{
% \centering Bandung, \makebox[0.5cm]{\hrulefill}/\makebox[0.5cm]{\hrulefill}/2013\\
\vspace{2cm} Nama: \makebox[3cm]{\hrulefill}\\ Pembimbing Tunggal
}
\end{document}

