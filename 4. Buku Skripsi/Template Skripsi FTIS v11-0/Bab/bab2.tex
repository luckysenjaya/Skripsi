%versi 2 (8-10-2016)
\chapter{Dasar Teori}
\label{chap:teori}

\section{Skripsi}
\label{sec:skripsi} 
Bab dua ini berisi dasar-dasar teori yang terkait dengan BPM, BPMN, BPMS, dan sistem email




\section{\textit{Business Process Management} (BPM)}
\label{sec:bpm}
\textit{Business Process} adalah kumpulan dari \textit{event}/kejadian, \textit{activity}/kegiatan, dan \textit{decision point}/keputusan serta melibatkan sejumlah aktor dan objek yang bertujuan untuk menghasilkan nilai dalam bentuk produk/jasa yang berguna bagi konsumen. Dari definisi proses bisnis, \textit{Business Process Management} dapat didefinisikan sebagai kumpulan metode, teknik, dan alat untuk menemukan, menganalisa, mendesain kembali, menjalankan, dan mengawasi proses bisnis. 

\subsection{Komponen \textit{Business Process}}
\label{sec:komponenBP}
\textit{Business Process Management} memiliki komponen-komponen sebagai berikut :
\begin{figure}[H]
	\centering
	\includegraphics[scale=0.5]{Gambar/Bab-2/1-bp-components}
	\caption{Komponen BPM} 
	\label{komponenbp}
\end{figure}

\begin{description}
	\item{\textit{Event}} \hfill \\\textit{Event} adalah kejadian yang terjadi saat proses bisnis berjalan. 
	\item{\textit{Activity}} \hfill \\\textit{Activity} adalah kumpulan kegiatan yang dapat dikerjakan. Ketika suatu \textit{Activity} berupa sebuah kegiatan yang sederhana, \textit{activity} disebut dengan \textit{task}. 
	\item{\textit{Decision Point}} \hfill \\\textit{Decision point} adalah keputusan yang mempengaruhi proses selanjutnya.
	\item{\textit{Actor}} \hfill \\ \textit{Actor} berupa individu, organisasi, maupun sistem yang mempengaruhi proses bisnis. 
	\item{\textit{Object}} \hfill \\ \textit{Object} dapat berupa objek fisik (peralatan, bahan baku, produk, dokumen) maupun non fisik (dokumen elektronik, basis data elektronik).
	\item{\textit{Positive/Negative Outcome}} \hfill \\ Hasil dari bisnis proses dapat menghasilkan nilai bagi konsumen (positif) atau tidak menghasilkan nilai (negatif).
\end{description}




\subsection{Siklus \textit{Business Process Management}}
\label{sec:siklusBPM}
Suatu proses bisnis tidak selalu berjalan dengan baik. Banyak hal yang tidak diantisipasi sebelumnya dapat menggangu proses bisnis. Untuk menjaga kualitas dari sebuah proses bisnis diperlukan pengawasan dan kontrol pada suatu fase tertentu serta perbaikan apabila diperlukan. Maka dari itu, suatu bisnis proses dapat dilihat sebagai suatu siklus yang terus menerus meningkatkan kualitasnya. Siklus dalam proses bisnis berupa :
\begin{figure}[H]
	\centering
	\includegraphics[scale=0.5]{Gambar/Bab-2/2-bpm-lifeCycle}
	\caption{Siklus BPM} 
	\label{siklusbpm}
\end{figure}

\begin{description}
	\item{\textit{Process Identification}} \hfill \\ Pada fase ini, suatu masalah bisnis ditemukan, kemudian proses-proses yang berhubungan dengan masalah bisnis tersebut diidentifikasi, dibatasi, dan dihubungkan satu sama lain. Proses ini terbagi menjadi dua tahap, yaitu \textit{designation} dan \textit{evaluation}. Tahap \textit{designation} bertujuan untuk mengenali proses-proses yang ada dan hubungan antar proses tersebut. Sedangkan tahap \textit{evaluation} memprioritaskan proses-proses yang menghasilkan nilai dan mempertimbangkan proses yang memiliki risiko atau tidak menghasilkan nilai. Fase ini menghasilkan arsitektur dari proses bisnis yang merepresentasikan proses bisnis dan relasi-relasinya.   
	\item{\textit{Process Discovery}} \hfill \\ Setiap proses yang relevan dengan masalah bisnis didokumentasikan, umumnya dalam bentuk model proses. Fase ini menghasilkan \textit{as-is process model}
	\item{\textit{Process Analysis}} \hfill \\ Pada fase ini, masalah pada model proses diidentifikasi, didokumentasikan, dan diukur kinerjanya dengan ukuran yang telah ditetapkan. Hasil dari fase ini adalah kumpulan masalah pada proses model.
	\item{\textit{Process Redesign}} \hfill \\ Tujuan dari fase ini adalah membuat perubahan pada proses yang dapat mengatasi berbagai kumpulan masalah yang telah diidentifikasi pada fase sebelumnya. Proses ini menghasilkan \textit{to-be process model}.
	\item{\textit{Process Implementation}} \hfill \\ Pada fase ini, model proses diimplementasikan untuk diekseskusi menggunakan \textit{Business Process Management System}.
	\item{\textit{Process Monitoring and Controlling}} \hfill \\ Setelah proses bisnis berjalan pada BPMS, berbagai data yang relevan dikumpulkan dan dianalisa untuk menentukan kualitas dari proses. Apabila terdapat masalah baru yang ditemukan, maka proses diulangi.
\end{description}



\section{\textit{Business Process Model Notation}}
\label{sec:bpmn}
Business Process Model Notation (BPMN) adalah notasi grafis yang menggambarkan langkah-langkah dalam proses bisnis. Notasi-notasi tersebut terdiri dari :

\subsection{\textit{Event}}
\label{sec:event}
Event merupakan kejadian yang terjadi pada proses bisnis yang dilambangkan dengan bentuk lingkaran. Notasi event secara umum terbagi menjadi tiga, yaitu \textit{start event, intermediate event,} dan \textit{end event}. \textit{Start event} menunjukkan dimulainya proses, \textit{intermediate event} dapat muncul ketika proses berjalan, sedangkan \textit{end event} menunjukkan berakhirnya proses.  
\begin{figure}[H]
	\centering
	\includegraphics[scale=1]{Gambar/Bab-2/bpmn/event1}
	\caption{Notasi \textit{Event}} 
	\label{event}
\end{figure}

\subsection{\textit{Activity}}
\label{sec:activity}

\
\begin{figure}[H]
	\centering
	\includegraphics[scale=1]{Gambar/Bab-2/bpmn/activity}
	\caption{Notasi \textit{Activity}} 
	\label{activity}
\end{figure}

\subsection{\textit{Gateway}}
\label{sec:gateway}
\textit{Gateway} merupakan simbol yang menentukan percabangan dan penggabungan jalur dalam proses. Gateway dilambangakan dengan belah ketupat. Beberapa macam adalah :
\begin{itemize}
	\item \textit{Exclusive Gateway} (XOR) berarti memilih salah satu dari cabang yang ada. 
	\item \textit{Inclusive Gateway} berarti memilih satu, beberapa, atau seluruh cabang yang ada.
	\item \textit{Parallel Gateway} berarti mengerjakan proses pada seluruh cabang yang ada.
	\item \textit{Event Based} berarti mengerjakan proses setelah suatu \textit{event} selesai.
\end{itemize} 
\begin{figure}[H]
	\centering
	\includegraphics[scale=1]{Gambar/Bab-2/bpmn/gateway}
	\caption{Notasi \textit{Gateway}} 
	\label{gateway}
\end{figure}


\subsection{\textit{Flow}}
\label{sec:flow}

\begin{figure}[H]
	\centering
	\includegraphics[scale=1]{Gambar/Bab-2/bpmn/flow}
	\caption{Notasi \textit{Flow}} 
	\label{flow}
\end{figure}

\subsection{\textit{Data}}
\label{sec:data}
\textit{Data Object} melambangkan informasi yang berjalan dalam proses seperti dokumen, e-mail, atau surat. Sedangkan \textit{Data Store} merupakan tempat proses membaca atau menyimpan data seperti basis data atau rak. 
\begin{figure}[H]
	\centering
	\includegraphics[scale=1]{Gambar/Bab-2/bpmn/data}
	\caption{Notasi \textit{Data}} 
	\label{data}
\end{figure}


\subsection{\textit{Artifact}}
\label{sec:artifacts}
\textit{Artifact} tidak mempengaruhi jalannya proses, tetapi hanya sebagai informasi tambahan agar proses lebih mudah dimengerti. Terdapat dua jenis, yaitu \textit{Text Annotation} dan \textit{Group}
\begin{figure}[H]
	\centering
	\includegraphics[scale=1]{Gambar/Bab-2/bpmn/artifact}
	\caption{Notasi \textit{Artifact} }
	\label{artifact}
\end{figure}


\subsection{\textit{Lanes} dan \textit{Pools}
\label{sec:poolslanes}

\begin{figure}[H]
	\centering
	\includegraphics[scale=1]{Gambar/Bab-2/bpmn/swimlane}
	\caption{Notasi \textit{Lanes & Pools}} 
	\label{lanespools}
\end{figure}


 

\section{\textit{Business Process Management System (BPMS)}}
\textit{Business Process Management System (BPMS)} adalah sistem yang mengkoordinasikan otomatisasi proses bisnis. Tujuan dari BPMS adalah menyelesaikan proses pada waktu yang ditentukan dan menggunakan sumber daya yang tepat. 

\subsection{Arsitektur BPMS}
Komponen-komponen BPMS beserta hubungannya yang ditunjukkan pada Gambar ~\ref{fig:arsitekturbpms} terdiri dari :
\begin{itemize}
	\item \textit{Execution Engine}, menyediakan beberapa fungsi seperti mengeksekusi proses, mendistribusikan \textit{task}, mengambil dan menyimpan data yang diperlukan. 
	\item \textit{Process Modeling Tool}, \textit{tool} untuk membuat model proses.
	\item \textit{Worklist Handler}, me
	\item \textit{Administration & Monitoring Tool}
	\item \textit{Repository}
	\item \textit{Execution Logs}
\end{itemize}
\begin{figure}[H]
	\centering
	\includegraphics[scale=0.5]{Gambar/Bab-2/bpms/bpms}
	\caption{Arsitektur BPMS} 
	\label{fig:arsitekturbpms}
\end{figure}



\section{Camunda}
Camunda adalah \textit{framework} BPMS berbasis Java yang mendukung \textit{workflow} BPMN dan otomatisasi proses bisnis. 

\subsection{Arsitektur BPMS Camunda}
Camunda memiliki komponen-komponen yang sama dengan BPMS. 
\begin{figure}[H]
	\centering
	\includegraphics[scale=0.5]{Gambar/Bab-2/bpms/arsitektur-camunda}
	\caption{Arsitektur BPMS Camunda} 
	\label{fig:arsitekturcamunda}
\end{figure}


\section{Email}







		
	

 
\subsection{Kutipan}
\label{subs:kutipan} 
Berikut contoh kutipan dari berbagai sumber, untuk keterangan lebih lengkap, silahkan membaca file referensi.bib yang disediakan juga di template ini.
Contoh kutipan:
\begin{itemize}
	\item Buku:~\cite{berg:08:compgeom} 
	\item Bab dalam buku:~\cite{kreveld:04:GIS}
	\item Artikel dari Jurnal:~\cite{buchin:13:median}
	\item Artikel dari prosiding seminar/konferensi:~\cite{kreveld:11:median}
	\item Skripsi/Thesis/Disertasi:~\cite{lionov:02:animasi}~\cite{wiratma:10:following}~\cite{wiratma:22:later}
	\item Technical/Scientific Report:~\cite{kreveld:07:watertight}
	\item RFC (Request For Comments):~\cite{RFC1654}
	\item Technical Documentation/Technical Manual:~\cite{Z.500}~\cite{unicode:16:stdv9}~\cite{google:16:and7}
	\item Paten:~\cite{webb:12:comm}
	\item Tidak dipublikasikan:~\cite{wiratma:09:median}~\cite{lionov:11:cpoly}
	\item Laman web:~\cite{erickson:03:cgmodel}  
	\item Lain-lain:~\cite{agung:12:tango}
\end{itemize}    
  
\subsection{Gambar}

Pada hampir semua editor, penempatan gambar di dalam dokumen \LaTeX{} tidak dapat dilakukan melalui proses {\it drag and drop}.
Perhatikan contoh pada file bab2.tex untuk melihat bagaimana cara menempatkan gambar.
Beberapa hal yang harus diperhatikan pada saat menempatkan gambar:
\begin{itemize}
	\item Setiap gambar {\bf harus} diacu di dalam teks (gunakan {\it field} {\sc label})
	\item {\it Field} {\sc caption} digunakan untuk teks pengantar pada gambar. Terdapat dua bagian yaitu yang ada di antara tanda $[$ dan $]$ dan yang ada di antara tanda $\{$ dan $\}$. Yang pertama akan muncul di Daftar Gambar, sedangkan yang kedua akan muncul di teks pengantar gambar. Untuk skripsi ini, samakan isi keduanya.
	\item Jenis file yang dapat digunakan sebagai gambar cukup banyak, tetapi yang paling populer adalah tipe {\sc png} (lihat Gambar~\ref{fig:ularpng}), tipe {\sc jpg} (Gambar~\ref{fig:ularjpg}) dan tipe {\sc pdf} (Gambar~\ref{fig:ularpdf})
	\item Besarnya gambar dapat diatur dengan {\it field} {\sc scale}.
	\item Penempatan gambar diatur menggunakan {\it placement specifier} (di antara tanda  $[$ dan $]$ setelah deklarasi gambar.
	Yang umum digunakan adalah {\bf H} untuk menempatkan gambar {\bf sesuai} penempatannya di file .tex atau  {\bf h} yang berarti "kira-kira" di sini. \\
	Jika tidak menggunakan {\it placement specifier}, \LaTeX{} akan menempatkan gambar secara otomatis untuk menghindari bagian kosong pada dokumen anda.
	Walaupun cara ini sangat mudah, hindarkan terjadinya penempatan dua gambar secara berurutan. 	
	\begin{itemize}
		\item Gambar~\ref{fig:ularpng} ditempatkan di bagian atas halaman, walaupun penempatannya dilakukan setelah penulisan 3 paragraf setelah penjelasan ini.
		\item Gambar~\ref{fig:ularjpg} dengan skala 0.5 ditempatkan di antara dua buah paragraf. Perhatikan penulisannya di dalam file bab2.tex!
		\item Gambar~\ref{fig:ularpdf} ditempatkan menggunakan {\it specifier} {\bf h}.
	\end{itemize}
\end{itemize}
 
\kant[14-15]
\begin{figure} 
	\centering  
	\includegraphics[scale=1]{ular-png}  
	\caption[Gambar {\it Serpentes} dalam format png]{Gambar {\it Serpentes} dalam format png} 
	\label{fig:ularpng} 
\end{figure} 

\kant[16]
\begin{figure}[H]
	\centering  
	\includegraphics[scale=0.5]{ular-jpg}  
	\caption[Ular kecil]{Ular kecil} 
	\label{fig:ularjpg} 
\end{figure} 
\kant[17-19]

\begin{figure}[h] 
	\centering  
	\includegraphics[scale=1]{ular-pdf}  
	\caption[ {\it Serpentes} betina]{ {\it Serpentes} jantan} 
	\label{fig:ularpdf} 
\end{figure} 
 
