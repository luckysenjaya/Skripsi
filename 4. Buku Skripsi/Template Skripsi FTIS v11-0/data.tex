%_____________________________________________________________________________
%=============================================================================
% data.tex v8 (02-10-2016) \ldots dibuat oleh Lionov - Informatika FTIS UNPAR
%
% Perubahan pada versi 8 (02-10-2016)
%	- Perubahan keterangan pada spacing: Otomatis spasi 1 untuk buku skripsi 
%	  final dan 1.5 untuk buku sidang
%	- Penggunaan kantlipsum
%_____________________________________________________________________________
%=============================================================================

%=============================================================================
% 								PETUNJUK
%=============================================================================
% Ini adalah file data (data.tex)
% Masukkan ke dalam file ini, data-data yang diperlukan oleh template ini
% Cara memasukkan data dijelaskan di setiap bagian
% Data yang WAJIB dan HARUS diisi dengan baik dan benar adalah SELURUHNYA !!
% Hilangkan tanda << dan >> jika anda menemukannya
%=============================================================================

%_____________________________________________________________________________
%=============================================================================
% 								BAGIAN 0
%=============================================================================
% PERHATIAN!! PERHATIAN!! Bagian ini hanya ada untuk sementara saja
% Jika "DAFTAR ISI" tidak bisa berada di bagian tengah halaman, isi dengan XXX
% jika sudah benar posisinya, biarkan kosong (i.e. \daftarIsiError{ })
%=============================================================================
\daftarIsiError{ }
%=============================================================================

%_____________________________________________________________________________
%=============================================================================
% 								BAGIAN I
%=============================================================================
% Tambahkan package2 lain yang anda butuhkan di sini
%=============================================================================
\usepackage{booktabs} 
\usepackage[table]{xcolor}
\usepackage{longtable}
\usepackage{amssymb}
\usepackage{todo}
\usepackage{verbatim} 		%multilne comment
\usepackage{pgfplots}
%=============================================================================

%_____________________________________________________________________________
%=============================================================================
% 								BAGIAN II
%=============================================================================
% Mode dokumen: menetukan halaman depan dari dokumen, apakah harus mengandung 
% prakata/pernyataan/abstrak dll (termasuk daftar gambar/tabel/isi) ?
% - kosong : tidak ada halaman depan sama sekali (untuk dokumen yang 
%            dipergunakan pada proses bimbingan)
% - cover : cover saja tanpa daftar isi, gambar dan tabel
% - sidang : cover, daftar isi, gambar, tabel 
% - sidang_akhir : mode sidang + abstrak + abstract
% - final : seluruh halaman awal dokumen (untuk cetak final)
% Jika tidak ingin mencetak daftar tabel/gambar (misalkan karena tidak ada 
% isinya), edit manual di baris 439 dan 440 pada file main.tex
%=============================================================================
% \mode{kosong}
% \mode{cover}
% \mode{sidang}
% \mode{sidang_akhir}
 \mode{final} 
%=============================================================================

%_____________________________________________________________________________
%=============================================================================
% 								BAGIAN III
%=============================================================================
% Line numbering: penomoran setiap baris, otomatis di-reset setiap berganti
% halaman
% - yes: setiap baris diberi nomor
% - no : baris tidak diberi nomor, otomatis untuk mode final
%=============================================================================
\linenumber{no}
%=============================================================================

%_____________________________________________________________________________
%=============================================================================
% 								BAGIAN IV
%=============================================================================
% Linespacing: jarak antara baris 
% - single	: wajib (dan otomatis jika ingin mencetak buku skripsi, opsi yang 
%			  disediakan untuk bimbingan, jika pembimbing tidak keberatan 
%			  (untuk menghemat kertas)
% - onehalf	: default dan wajib (dan otomatis) jika ingin mencetak dokumen
%             untuk sidang.
% - double 	: jarak yang lebih lebar lagi, jika pembimbing berniat memberi 
%             catatan yg banyak di antara baris (dianjurkan untuk bimbingan)
%=============================================================================
\linespacing{single}
%\linespacing{onehalf}
%\linespacing{double}
%=============================================================================

%_____________________________________________________________________________
%=============================================================================
% 								BAGIAN V
%=============================================================================
% Tidak semua skripsi memuat gambar dan/atau tabel. Untuk skripsi yang seperti
% itu, tidak diperlukan Daftar Gambar dan Daftar Tabel. Sayangnya hal ini 
% sulit dilakukan secara manual karena membutuhkan kedisiplinan pengguna 
% template.  
% Jika tidak akan menampilkan Daftar Gambar/Tabel, isi dengan NO. Jika ingin
% menampilkan, kosongkan parameter (i.e. \gambar{ }, \tabel{ })
%=============================================================================
\gambar{ }
\tabel{NO}
%=============================================================================

%_____________________________________________________________________________
%=============================================================================
% 								BAGIAN VI
%=============================================================================
% Bab yang akan dicetak: isi dengan angka 1,2,3 s.d 9, sehingga bisa digunakan
% untuk mencetak hanya 1 atau beberapa bab saja
% Jika lebih dari 1 bab, pisahkan dengan ',', bab akan dicetak terurut sesuai 
% urutan bab (e.g. \bab{1,2,3}).
% Untuk mencetak seluruh bab, kosongkan parameter (i.e. \bab{ })  
% Catatan: Jika ingin menambahkan bab ke-10 dan seterusnya, harus dilakukan 
% secara manual
%=============================================================================
\bab{ }
%=============================================================================

%_____________________________________________________________________________
%=============================================================================
% 								BAGIAN VII
%=============================================================================
% Lampiran yang akan dicetak: isi dengan huruf A,B,C s.d I, sehingga bisa 
% digunakan untuk mencetak hanya 1 atau beberapa lampiran saja
% Jika lebih dari 1 lampiran, pisahkan dengan ',', lampiran akan dicetak 
% terurut sesuai urutan lampiran (e.g. \bab{A,B,C}).
% Jika tidak ingin mencetak lampiran apapun, isi dengan -1 (i.e. \lampiran{-1})
% Untuk mencetak seluruh mapiran, kosongkan parameter (i.e. \lampiran{ })  
% Catatan: Jika ingin menambahkan lampiran ke-J dan seterusnya, harus 
% dilakukan secara manual
%=============================================================================
\lampiran{ }
%=============================================================================

%_____________________________________________________________________________
%=============================================================================
% 								BAGIAN VIII
%=============================================================================
% Data diri dan skripsi/tugas akhir
% - namanpm: Nama dan NPM anda, penggunaan huruf besar untuk nama harus benar
%			 dan gunakan 10 digit npm UNPAR, PASTIKAN BAHWA BENAR !!!
%			 (e.g. \namanpm{Jane Doe}{1992710001}
% - judul : Dalam bahasa Indonesia, perhatikan penggunaan huruf besar, judul
%			tidak menggunakan huruf besar seluruhnya !!! 
% - tanggal : isi dengan {tangga}{bulan}{tahun} dalam angka numerik, jangan 
%			  menuliskan kata (e.g. AGUSTUS) dalam isian bulan
%			  Tanggal ini adalah tanggal dimana anda akan melaksanakan sidang 
%			  ujian akhir skripsi/tugas akhir
% - pembimbing: isi dengan pembimbing anda, lihat daftar dosen di file dosen.tex
%				jika pembimbing hanya 1, kosongkan parameter kedua 
%				(e.g. \pembimbing{\JND}{  } ) , \JND adalah kode dosen
% - penguji : isi dengan para penguji anda, lihat daftar dosen di file dosen.tex
%				(e.g. \penguji{\JHD}{\JCD} ) , \JND dan \JCD adalah kode dosen
% !!Lihat singkatan pembimbing dan penguji anda di file dosen.tex
%=============================================================================
\namanpm{Lucky Senjaya Darmawan}{2012730009}	%hilangkan tanda << & >>
\tanggal{30}{5}{2017}			%hilangkan tanda << & >>
\pembimbing{\GDK}{ } %hilangkan tanda << & >>    
\penguji{\VSM}{\RDL} 				%hilangkan tanda << & >>
%=============================================================================

%_____________________________________________________________________________
%=============================================================================
% 								BAGIAN IX
%=============================================================================
% Judul dan title : judul bhs indonesia dan inggris
% - judulINA: judul dalam bahasa indonesia
% - judulENG: title in english
% PERHATIAN: - langsung mulai setelah '{' awal, jangan mulai menulis di baris 
%			   bawahnya
%			 - Gunakan \texorpdfstring{\\}{} untuk pindah ke baris baru
%			 - Judul TIDAK ditulis dengan menggunakan huruf besar seluruhnya !!
%			 - Gunakan perintah \texorpdfstring{\\}{} untuk baris baru
%=============================================================================
\judulINA{Studi dan Integrasi \textit{Workflow} menggunakan BPMS dan Sistem Email}
\judulENG{Workflow Study and Integration using BPMS and Email System}
%_____________________________________________________________________________
%=============================================================================
% 								BAGIAN X
%=============================================================================
% Abstrak dan abstract : abstrak bhs indonesia dan inggris
% - abstrakINA: abstrak bahasa indonesia
% - abstrakENG: abstract in english
% PERHATIAN: langsung mulai setelah '{' awal, jangan mulai menulis di baris 
%			 bawahnya
%=============================================================================
\abstrakINA{\textit{Workflow} merupakan pemodelan proses bisnis yang dapat digambarkan sebagai \textit{flow map} atau BPMN \textit{(Business Process Model and Notation).} \textit{Workflow} ini dapat diotomasi menggunakan perangkat lunak untuk otomasi proses bisnis, yaitu BPMS \textit{(Business Process Management System)}, seperti Camunda. Agar eksekusi \textit{workflow} lebih alamiah dan sesuai dengan model komunikasi organisasi saat ini, maka \textit{event} dapat dipropagasi dan diintegrasikan dengan sistem email. Tetapi mekanisme propagasi ini belum tersedia di BPMS Camunda.

Dalam skripsi ini, dikembangkan suatu mekanisme propagasi email BPMS menggunakan sistem Java. Integrasi sistem email pada BPMS dibuat pada \textit{event} yang tertera pada \textit{user task}. \textit{User task} adalah suatu tugas yang perlu dilakukan oleh pengguna. Pada \textit{task} tersebut disisipkan implementasi sistem email menggunakan Java. Ketika ada suatu \textit{user task}, sistem email akan mengirim email ke pengguna yang akan mengerjakan task tersebut. Email tersebut berisi tautan yang mengarah ke tugas yang perlu dikerjakan tersebut.  

Berdasarkan pengujian menggunakan kasus Pengajuan Proposal dan Pendaftaran BPJS, sistem dapat mengirim email ke masing-masing pemilik \textit{task}. Email langsung dikirim setelah \textit{user task} siap untuk dikerjakan. Dengan demikian dapat disimpulkan bahwa integrasi \textit{workflow} dengan mempropagasi \textit{task event} dengan sistem email pada BPMS Camunda telah berhasil dikembangkan}

\abstrakENG{Workflow is business process model that can be described as a flow map or BPMN (Business Process Model and Notation). Workflow can be automated using BPMS (Business Process Management System), such as Camunda. Workflow execution will be more natural with current organizational communication models, event can be propagated and integrated with email system. But this mechanism hasn't available in BPMS Camunda.


This thesis will develop BPMS email propagation using Java system. Email system integration with BPMS placed inside user task event. User task is task that need to be done by the user. Email system implementation using Java attached to the user task. When there is a user task, email system will send email to user. The email contains link to the task that needs to be done. 

According to experiment Pengajuan Proposal and Pendaftaran BPJS, system can send email to task owner. System send the email instantly after task ready. With this, we can conclude that workflow integration by propagating event task with email system on BPMS Camunda has been successfully developed.
} 
%=============================================================================

%_____________________________________________________________________________
%=============================================================================
% 								BAGIAN XI
%=============================================================================
% Kata-kata kunci dan keywords : diletakkan di bawah abstrak (ina dan eng)
% - kunciINA: kata-kata kunci dalam bahasa indonesia
% - kunciENG: keywords in english
%=============================================================================
\kunciINA{Proses Bisnis, BPMN, BPMS, Camunda, Email, Integrasi Email}
\kunciENG{Business Process, BPMN, BPMS, Camunda, Email, Email Integration}
%=============================================================================

%_____________________________________________________________________________
%=============================================================================
% 								BAGIAN XII
%=============================================================================
% Persembahan : kepada siapa anda mempersembahkan skripsi ini ...
%=============================================================================
\untuk{Teknik Informatika UNPAR dan diri sendiri}
%=============================================================================

%_____________________________________________________________________________
%=============================================================================
% 								BAGIAN XIII
%=============================================================================
% Kata Pengantar: tempat anda menuliskan kata pengantar dan ucapan terima 
% kasih kepada yang telah membantu anda bla bla bla ....  
%=============================================================================
\prakata{Puji syukur kepada Tuhan Yang Maha ESA atas seluruh karunia-Nya yang diberikan kepada penulis sehingga dapat menyelesaikan tugas akhir ini dengan baik. Tugas akhir berjudul \textbf{Studi dan Integrasi Workflow Menggunakan BPMS dan Sistem Email} diajukan sebagai salah satu syarat untuk menyelesaikan studi pada jurusan Teknik Informatika di Fakultas Teknologi Informasi dan Sains, Universitas Katolik Parahyangan. Pada kesempatan ini, penulis juga mengucapkan terima kasih kepada pihak-pihak yang telah memberikan dukungan untuk menyelesaikan tugas akhir ini, yaitu:
\begin{itemize}
	\item Orang tua dan keluarga yang selalu memberikan dukungan selama proses pengerjaan tugas akhir ini.
	\item Bapak Gede Karya, M.T.,CISA sebagai pembimbing utama yang telah meluangkan waktunya untuk membantu dan membimbing dengan penuh kesabaran selama pengerjaan tugas akhir ini.
	\item Dr. Veronica Sri Moertini dan Ibu Rosa De Lima, M.KOM. sebagai penguji yang telah meluangkan waktunya untuk menguji dan memberikan masukan-masukan untuk tugas akhir ini.
	\item Teman-teman Teknik Informatika UNPAR angkatan 2012 yang telah memberikan semangat untuk menyelesaikan tugas akhir ini.
	\item Pihak-pihak lain yang belum disebutkan yang berperan dalam penyelesaian tugas akhir ini.
	
	Akhir kata, penulis berharap tugas akhir ini dapat bermanfaat bagi para pembaca yang hendak melakukan penelitian dan pengembangan yang terkait dengan tugas akhir ini. Penulis menyadari bahwa penyusunan tugas akhir ini masih jauh dari semburna, sehingga penulis menerima kritik dan saran yang bersifat membangun.
\end{itemize}

  }
%=============================================================================

%_____________________________________________________________________________
%=============================================================================
% 								BAGIAN XIV
%=============================================================================
% Tambahkan hyphen (pemenggalan kata) yang anda butuhkan di sini 
%=============================================================================
\hyphenation{ma-te-ma-ti-ka}
\hyphenation{fi-si-ka}
\hyphenation{tek-nik}
\hyphenation{in-for-ma-ti-ka}
%=============================================================================

%_____________________________________________________________________________
%=============================================================================
% 								BAGIAN XV
%=============================================================================
% Tambahkan perintah yang anda buat sendiri di sini 
%=============================================================================
\newcommand{\vtemplateauthor}{lionov}
\pgfplotsset{compat=newest}
\usetikzlibrary{patterns}
%=============================================================================

% Copyright \textcopyright [Lionov] [09-10-2016]. All rights reserved