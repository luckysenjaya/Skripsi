\chapter{Kode Skenario}
\label{lamp:kodeskenario}

\section{Kasus 1 - Pengajuan Proposal}
\label{sec:kodekasus1}
\begin{lstlisting}[language=Java,basicstyle=\tiny,caption=PengajuanProposal.java]
package pengajuanproposal;

import org.camunda.bpm.application.ProcessApplication;
import org.camunda.bpm.application.impl.ServletProcessApplication;

@ProcessApplication("PengajuanProposal App")
public class PengajuanProposal extends ServletProcessApplication{

}
\end{lstlisting}

\begin{lstlisting}[language=html,basicstyle=\tiny,caption=MengunggahProposal.html]
<html>
<head>
<body>
	<form method="post" name="upload-dokumen">
		<input type="file"
       cam-variable-name="proposal"
       cam-variable-type="File"
       cam-max-filesize="10000000"/>
	</form>
</body>
</html>
\end{lstlisting}

\begin{lstlisting}[language=html,basicstyle=\tiny,caption=MemeriksaProposal.html]
<html>
<head></head>

<body>
<form role="form" name="form">
  	<a cam-file-download="proposal">Download Dokumen</a>


    <p>Apakah Proposal layak?</p>
    <input cam-variable-name="valid"
             cam-variable-type="Boolean"
             type="checkbox"
             name="valid"
             class="form-control" />
    
</form> 
</body>
</html>
\end{lstlisting}


\begin{lstlisting}[language=html,basicstyle=\tiny,caption=MelihatStatusProposal.html]
<html>
	<head></head>
	<body>
	<h> Proposal sudah diterima </h>
	
	<form role="form" name="form">
  	<a cam-file-download="proposal">Lihat Proposal</a>
  	</form>
	


</html>
\end{lstlisting}



\section{Kasus 2 - Pendaftaran BPJS}
\label{sec:kode_kasus2}
\begin{lstlisting}[language=Java,basicstyle=\tiny,caption=PendaftaranBPJS.java]
package pengajuanproposal;

package org.camunda.bpm.getstarted.pendaftaranbpjs;

import org.camunda.bpm.application.ProcessApplication;
import org.camunda.bpm.application.impl.ServletProcessApplication;

@ProcessApplication("PendaftaranBPJS App")
public class PendaftaranBPJS extends ServletProcessApplication {
	
}

\end{lstlisting}

\begin{lstlisting}[language=Java,basicstyle=\tiny,caption=PembangkitBarcode.java]
package org.camunda.bpm.getstarted.pendaftaranbpjs;

import java.util.Random;

import org.camunda.bpm.engine.delegate.DelegateExecution;
import org.camunda.bpm.engine.delegate.JavaDelegate;

public class PembangkitBarcode implements JavaDelegate{

	public int getBarcode(){
		Random rand = new Random();
		int nomor = rand.nextInt(10000000)+100000;
		return nomor;
	}
	
	public void execute(DelegateExecution execution) throws Exception {
		String barcode = ""+this.getBarcode();
		execution.setVariable("barcode", barcode);
	}
}

\end{lstlisting}

\begin{lstlisting}[language=Java,basicstyle=\tiny,caption=PembangkitJadwal.java]
package org.camunda.bpm.getstarted.pendaftaranbpjs;

import java.util.Random;
import java.util.logging.Logger;

import org.camunda.bpm.engine.delegate.DelegateExecution;
import org.camunda.bpm.engine.delegate.JavaDelegate;
import org.camunda.bpm.engine.runtime.ProcessInstance;

public class PembangkitJadwal implements JavaDelegate{
	public final static Logger LOGGER = Logger.getLogger("pembangkit-jadwal");
	public String jadwalKedatangan(Object hari){
		
		return hari+"";
	}
	
	public int nomorAntrian(){
		Random rand = new Random();
		int nomor = rand.nextInt(10);
		return nomor;
	}
	
	public void execute(DelegateExecution execution) throws Exception {
		execution.getVariable("jadwalHari");
		String jadwal = this.jadwalKedatangan(execution.getVariable("jadwalHari"));
		int nomor = this.nomorAntrian();

		execution.setVariable("jadwal", jadwal);
		execution.setVariable("nomor", nomor);

	}
}

\end{lstlisting}

\begin{lstlisting}[language=Java,basicstyle=\tiny,caption=PembangkitNomor.java]
package org.camunda.bpm.getstarted.pendaftaranbpjs;

import java.util.Random;

import java.util.logging.Logger;

import org.camunda.bpm.engine.delegate.DelegateExecution;
import org.camunda.bpm.engine.delegate.JavaDelegate;

public class PembangkitNomor implements JavaDelegate{
	public final static Logger LOGGER = Logger.getLogger("pendaftaran-bpjs");
	int nomor;
	int uangDaftar;
	
	public int nomorPembayaran(){
		
		Random rand = new Random();
		nomor = rand.nextInt(10)+1;
		
		return nomor;
	}
	
	public int uangPendaftaran(){
		uangDaftar = 50000;
		return uangDaftar;
	}

	public void execute(DelegateExecution execution) throws Exception {

		
		this.nomorPembayaran();
		this.uangPendaftaran();
		execution.setVariable("uangDaftar", uangDaftar);
		execution.setVariable("nomor", nomor);
	
	}

}

\end{lstlisting}


\begin{lstlisting}[language=html,basicstyle=\tiny,caption=pendaftaran-bpjs.html]
<html>
<head><title>Pendaftaran BPJS</title></head>
<body>
	<form name="pendaftaranBPJS" role="form">
	  <h3>Pendaftaran BPJS</h3>
	  
	  <div class="control-group">
		<label class="control-label" for="nama">Nama</label>
	    <div class="controls">
	      <input id="nama"
	             class="form-control"
	             cam-variable-name = "nama"
	             cam-variable-type = "String"
	             type="text"
	             required>
	    </div>	
		<label class="control-label" for="tanggalLahir">Tanggal Lahir</label>
	    <div class="controls">
	      <input id="tanggalLahir"
	             class="form-control"
	             cam-variable-name = "tanggalLahir"
	             cam-variable-type = "String"
	             type="text"
	             required>
	    </div>		
		<label class="control-label" for="nik">NIK</label>
	    <div class="controls">
	      <input id="nik"
	             class="form-control"
	             cam-variable-name = "nik"
	             cam-variable-type = "String"
	             type="text"
	             required>
	    </div>		
		<label class="control-label" for="kelas">Kelas</label>
	    <div class="controls">
	      <input id="kelas"
	             class="form-control"
	             cam-variable-name = "kelas"
	             cam-variable-type = "String"
	             type="text"
	             required>
	    </div>
	    

	</form>

</body>
</html>
\end{lstlisting}

\begin{lstlisting}[language=html,basicstyle=\tiny,caption=upload-dokumen.html]
<html>
<head></head>

<body>
	<form method="post" name="upload-dokumen">
		<input type="file"
       cam-variable-name="INVOICE_DOCUMENT"
       cam-variable-type="File"
       cam-max-filesize="10000000" />
	</form>
</body>
</html>
\end{lstlisting}

\begin{lstlisting}[language=html,basicstyle=\tiny,caption=pilih-jadwal.html]
<html>

<head></head>

<body>
	<p>Halaman pilih jadwal verifikasi dokumen.</p>
	<form>
		<select 
			cam-variable-name="jadwalHari"
			cam-variable-type="String"
	        cam-choices="jadwalHariPilihan"
	        >
		  		<option value="senin">Senin</option>
		  		<option value="selasa">Selasa</option>
		  		<option value="rabu">Rabu</option>
		  		<option value="kamis">Kamis</option>
		  		<option value="jumat">Jumat</option>
		</select>
	</form>


</body>
</html>
\end{lstlisting}

\begin{lstlisting}[language=html,basicstyle=\tiny,caption=nomor-pembayaran.html]
<html>

<head></head>


	
<body>
<form role="form" name="form">
	<script cam-script type="text/form-script">
    	camForm.on('form-loaded', function() {
	      camForm.variableManager.fetchVariable('uangDaftar');
	      camForm.variableManager.fetchVariable('nomor');
	      
    });
    	camForm.on('variables-restored', function() {
	      $scope.uangDaftar = camForm.variableManager.variableValue('uangDaftar');
	      $scope.nomor = camForm.variableManager.variableValue('nomor');
	      
    });
  	</script>
  	<table>

    <tr>
      <td>Uang Daftar:</td>
      <td>{{ uangDaftar }}</td>
    </tr>

    <tr>
      <td>Nomor:</td>
      <td>{{nomor}}</td>
    </tr>
    </table>
</form> 
</body>
</html>




\end{lstlisting}

\begin{lstlisting}[language=html,basicstyle=\tiny,caption=ringkasan-jadwal.html]
<html>

<head></head>
<script>
	function printDiv(divName){
		var printContents = document.getElementById(divName).innerHTML;
		var originalContents = document.body.innerHTML;
		
		document.body.innerHTML = printContents;
		
		window.print();
		
		document.body.innerHTML = originalContents;
		window.close();
	
	
	}


</script>

	
<body>
<form role="form" name="form">
	<script cam-script type="text/form-script">
    	camForm.on('form-loaded', function() {
	      camForm.variableManager.fetchVariable('nomor');
	      camForm.variableManager.fetchVariable('jadwal');
	      
    });
    	camForm.on('variables-restored', function() {
	      $scope.nomor = camForm.variableManager.variableValue('nomor');
	      $scope.jadwal = camForm.variableManager.variableValue('jadwal');
	      
    });
  	</script>
  <div id ="printableArea">
  	<table>

    <tr>
      <td>Nomor :</td>
      <td>{{ nomor }}</td>
    </tr>

    <tr>
      <td>jadwal :</td>
      <td>{{jadwal}}</td>
    </tr>
    
    </table>
    </div>
</form> 
 <input type ="button" onclick = "printDiv('printableArea')" value = "Cetak Jadwal"/>
</body>
</html>
\end{lstlisting}

\begin{lstlisting}[language=html,basicstyle=\tiny,caption=ringkasa-pembayaran.html]
<html>

<head></head>


	
<body>
<form role="form" name="form">
	<script cam-script type="text/form-script">
    	camForm.on('form-loaded', function() {
	      camForm.variableManager.fetchVariable('nomor');
	      camForm.variableManager.fetchVariable('uangDaftar');
	      
    });
    	camForm.on('variables-restored', function() {
	      $scope.nomor = camForm.variableManager.variableValue('nomor');
	      $scope.uangDaftar = camForm.variableManager.variableValue('uangDaftar');
	      
    });
  	</script>
  	<table>

    <tr>
      <td>Nomor :</td>
      <td>{{ nomor }}</td>
    </tr>

    <tr>
      <td>Uang Daftar:</td>
      <td>{{uangDaftar}}</td>
    </tr>
    </table>
</form> 

</body>
</html>
\end{lstlisting}

\begin{lstlisting}[language=html,basicstyle=\tiny,caption=verifikasi-pendaftaran.html]
<html>

<head></head>


	
<body>
<form role="form" name="form">
	<script cam-script type="text/form-script">
    	camForm.on('form-loaded', function() {
	      camForm.variableManager.fetchVariable('nama');
	      camForm.variableManager.fetchVariable('tanggalLahir');
	      camForm.variableManager.fetchVariable('nik');
	      camForm.variableManager.fetchVariable('kelas');
	      camForm.variableManager.fetchVariable('jadwal');
	      camForm.variableManager.fetchVariable('nomor');
	      
    });
    	camForm.on('variables-restored', function() {
		$scope.nama = camForm.variableManager.variableValue('nama');
	    $scope.tanggalLahir = camForm.variableManager.variableValue('tanggalLahir');
	    $scope.nik = camForm.variableManager.variableValue('nik');
	    $scope.kelas = camForm.variableManager.variableValue('kelas');
	    $scope.nomor = camForm.variableManager.variableValue('nomor');
	    $scope.jadwal = camForm.variableManager.variableValue('jadwal');
	      
    });
  	</script>
  	<p>Download Dokumen Persyaratan</p>
  	<a cam-file-download="INVOICE_DOCUMENT">Download</a>
  	<table>
  	
  	

    <tr>
      <td>Nama:</td>
      <td>{{ nama }}</td>
    </tr>
    
    <tr>
      <td>Tanggal Lahir:</td>
      <td>{{ tanggalLahir }}</td>
    </tr>
    
    <tr>
      <td>Nik:</td>
      <td>{{ nik }}</td>
    </tr>
    
    <tr>
      <td>Kelas:</td>
      <td>{{ kelas }}</td>
    </tr>
    
    <tr>
      <td>Nomor:</td>
      <td>{{ nomor }}</td>
    </tr>

    <tr>
      <td>jadwal:</td>
      <td>{{jadwal}}</td>
    </tr>
    </table>
    
    <p>Apakah dokumen dan persyaratan lengkap?</p>
    <input cam-variable-name="valid"
             cam-variable-type="Boolean"
             type="checkbox"
             name="valid"
             class="form-control" />
    
</form> 
</body>
</html>
\end{lstlisting}

\begin{lstlisting}[language=html,basicstyle=\tiny,caption=cetak-kartu.html]
<html>

<head></head>
<script>
	function printDiv(divName){
		var printContents = document.getElementById(divName).innerHTML;
		var originalContents = document.body.innerHTML;
		
		document.body.innerHTML = printContents;
		
		window.print();
		
		document.body.innerHTML = originalContents;
		window.close();
	
	
	}


</script>
	
<body>
<form role="form" name="form">
	<script cam-script type="text/form-script">
    	camForm.on('form-loaded', function() {
	      camForm.variableManager.fetchVariable('nama');
	      camForm.variableManager.fetchVariable('tanggalLahir');
	      camForm.variableManager.fetchVariable('nik');
	      camForm.variableManager.fetchVariable('kelas');
	      camForm.variableManager.fetchVariable('barcode');
	      
    });
    	camForm.on('variables-restored', function() {
			$scope.nama = camForm.variableManager.variableValue('nama');
	    $scope.tanggalLahir = camForm.variableManager.variableValue('tanggalLahir');
	    $scope.nik = camForm.variableManager.variableValue('nik');
	    $scope.kelas = camForm.variableManager.variableValue('kelas');
	    $scope.barcode = camForm.variableManager.variableValue('barcode');
	      
    });
  	</script>
  	<div id ="printableArea">
  	<table>
  	


    <tr>
      <td>Nama:</td>
      <td>{{ nama }}</td>
    </tr>
    
    <tr>
      <td>Tanggal Lahir:</td>
      <td>{{ tanggalLahir }}</td>
    </tr>
    
    <tr>
      <td>Nik:</td>
      <td>{{ nik }}</td>
    </tr>
    
    <tr>
      <td>Kelas:</td>
      <td>{{ kelas }}</td>
    </tr>
    
    <tr>
      <td>Barcode:</td>
      <td>{{ barcode }}</td>
    </tr>

    </table>
    

</form> 
</div>
    <input type ="button" onclick = "printDiv('printableArea')" value = "Cetak Kartu"/>
</body>
</html>
\end{lstlisting}