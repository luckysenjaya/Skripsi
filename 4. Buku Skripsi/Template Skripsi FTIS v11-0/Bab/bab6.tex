\chapter{Kesimpulan dan Saran}
\label{chap:kesimpulan_saran}
Pada bab enam ini akan dijelaskan mengenai kesimpulan dan saran yang didapat dari propagasi sistem email dengan Camunda 
\section{Kesimpulan}
\label{sec:kesimpulan}
Berdasarkan hasil pengembangan propagasi sistem email dengan Camunda, didapatkan beberapa kesimpulan sebagai berikut :
\begin{enumerate}
	\item \textit{Workflow} dapat dimodelkan sebagai BPMN yang dapat divisualisasikan oleh BPMS. 
	\item \textit{Event-event} dapat dipropagasi via email sehingga aktor dapat mengetahui apabila ada \textit{task} yang harus dikerjakan. Dengan demikian akan meningkatkan efektifitas dan efisiensi proses bisnis.
	\item \textit Propagasi email dapat dilakukan dengan cara menyisipkan \textit{Task Listener} di event yang akan dipropagasi. Selain itu dibutuhkan peran admin untuk mendaftarkan alamat email aktor.
	\item \textit Pengujian telah dilakukan dengan dua skenario dan dapat berjalan dengan baik.
	

\end{enumerate}

\section{Saran}
\label{sec:saran}
Berdasarkan kesimpulan yang didapat, ada beberapa saran untuk penelitian dan pengembangan lebih lanjut, antara lain :
\begin{enumerate}
	\item Mekanisme propagasi email dapat dibuat dalam bentuk \textit{library} sehingga tidak perlu membuka kode dan cukup memasukkan alamat email dan password.
	\item Aspek integrasi bisa ditambahkan dengan \textit{external tasks}, yaitu sistem di luar Camunda dengan memanfaatkan \textit{web service}.
\end{enumerate}