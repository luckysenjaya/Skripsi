%versi 2 (8-10-2016) 
\chapter{Pendahuluan}
\label{chap:intro}
   
\section{Latar Belakang}
\label{sec:label}

\textit{Workflow} merupakan pemodelan proses bisnis yang dapat digambarkan sebagai \textit{flow map} atau BPMN \textit{(Business Process Model and Notation).} \textit{Workflow} ini dapat diotomasi menggunakan BPMS \textit{(Business Process Management System)}, seperti Camunda. Agar eksekusi \textit{workflow} lebih alamiah dengan model komunikasi organisasi saat ini, maka \textit{event} dapat dipropagasi dan diintegrasikan dengan sistem email. 
Dalam skripsi ini, akan dibuat suatu integrasi antara \textit{user task} dan sistem email. \textit{User task} adalah suatu tugas yang perlu dilakukan oleh \textit{user}. Ketika ada suatu \textit{user task}, sistem email akan mengirimkan email ke user yang akan mengerjakan task tersebut. Email tersebut akan berisi tautan yang mengarah ke tugas yang perlu dikerjakan tersebut.





\section{Rumusan Masalah}
\label{sec:rumusan}

Berdasarkan latar belakang yang dipaparkan sebelumnya, maka rumusan masalah dalam skripsi ini adalah sebagai berikut :
\begin{enumerate}
	\item Bagaimana memodelkan \textit{workflow} dengan BPMN?
	\item Event-event \textit{workflow} apa saja yang dapat dipropagasi ke sistem email?
	\item Bagaimana mekanisme propagasi dan integrasi \textit{workflow} dengan sistem email?
\end{enumerate} 




\section{Tujuan}
\label{sec:tujuan}

Berdasarkan rumusan masalah yang dipaparkan sebelumnya, tujuan dari penelitian ini adalah :
\begin{enumerate}
	\item Memodelkan \textit{workflow} dengan BPMN.
	\item Mengidentifikasi event-event \textit{workflow} yang dapat dipropagasi ke sistem email.
	\item Menentukan mekanisme propagasi dan mengintegrasikan \textit{workflow} dengan sistem email.
\end{enumerate}




\section{Batasan Masalah}
\label{sec:batasan}




\section{Metodologi}
\label{sec:metlit}

Metodologi yang digunakan dalam penelitian ini adalah sebagai berikut:
\begin{enumerate}
	\item Melakukan studi mengenai proses bisnis, \textit{workflow}, \textit{Business Process Model and Notation (BPMN)}, \textit{Business Process Management System (BPMS)}, dan sistem e-mail. 
	\item Memodelkan proses bisnis tertentu menggunakan BPMN.
	\item Mengidentifikasikan \textit{event-event} dari \textit{workflow} yang dapat diintegrasikan dengan sistem email.
	\item Merancang integrasi sistem email.
	\item Mengimplementasikan sistem email ke BPMS.
	\item Melakukan pengujian fungsionalitas.
\end{enumerate}




\section{Sistematika Pembahasan}
\label{sec:sispem}

\begin{enumerate}
	\item Bab 1 Pendahuluan, berisi latar belakang masalah, rumusan masalah, tujuan penelitian, batasan masalah, metodologi penelitian, dan sistematika penulisan.
	\item Bab 2 Dasar Teori, berisi dasar teori yang mencakup \textit{Business Process Management}, \textit{Business Process Model and Notation (BPMN)}, \textit{Business Process Management System (BPMS)}, dan sistem e-mail.
	\item Bab 3 Analisis, Berisi analisis Camunda dan analisis BPMN dengan menggunakan contoh kasus.
	\item Bab 4 Perancangan, Berisi rancangan algoritma yang akan dibuat.
	\item Bab 5 Implementasi, dan Pengujian Berisi implementasi dari program yang dibuat dan pengujian aplikasi berdasarkan contoh kasus pada bab 3.
	\item Bab 6 Penutup, Berisi kesimpulan dan saran-saran untuk pengembangan selanjutnya.
\end{enumerate}


